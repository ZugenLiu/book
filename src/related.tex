%	$Id: related.tex,v 1.28 2012/08/20 09:45:12 pooka Exp $

\section{Related Work}
\label{chap:related}

This chapter surveys and compares work related to the anykernel
architecture and rump kernels.

\subsection{Running Kernel Code in Userspace}

Operating systems running in userspace~\cite{dike:uml,eco:vkernel}
make it possible to run an entire monolithic operating system inside
a userspace process.  At a basic level this approach shares the
same drawbacks for our purposes as any other full system virtualization
solution.  The use of the host OS as the VMM can be both a simplifying
factor and a complicating one.  It simplifies things because no
separate VMM is required; the host OS is enough.  On the other
hand, it complicates things because there is no portability layer.
For example, the Dragonfly vkernel~\cite{eco:vkernel} relies heavily
on host kernel interfaces to be able to create virtual memory spaces
for processes of the guest OS.  A usermode OS port has overlap with
a rump kernel.  For example, the virtual I/O drivers described in
Section~\ref{sect:iobackend} can be used equally in a rump kernel
and a usermode OS.  The differences stem from the fact that while
a usermode OS ports the entire OS as-is on top of the host, a rump
kernel adapts the OS codebase so that it is both componentized and
can relegate functionality directly to the host.

The Alpine~\cite{ely:alpine} network protocol development infrastructure
provides an environment for running unmodified FreeBSD 3.3 kernel TCP/IP
code in userspace without requiring full virtualization of the source OS.
Alpine is implemented before the system call layer by overriding libc
with the TCP/IP stack itself run in application process context.  It can
be viewed as an early and domain specific version of a rump kernel.

Rialto~\cite{draves:unifying} is an operating system with a unified
interface both for userspace and the kernel making it possible to run
most code in either environment.  Rialto was designed and implemented from
ground-up as opposed to our approach of starting with an existing system.
Interesting ideas include the definition of both internal and external
linkage for an interface.

The Linux Kernel Library~\cite{purdila:lkl} (LKL) provides the Linux
kernel as a monolithic library to be used with 3rd party code such
as applications.  It is implemented as a separate architecture for
Linux which is configured and compiled to produce the library
kernel.  This approach contrasts the rump kernel approach where the
functionality consists of multiple components which are combined by
the linker to produce the final result.  LKL uses many of the same
basic approaches, such as relying on host threads.  However, in
contrast to our lightweight CPU scheduler, the Linux scheduler is
involved with scheduling in LKL, so there are two layers of
schedulers at runtime.

\subsection{Microkernel Operating Systems}

Microkernel operating
systems~\cite{accetta:mach,herder:minix3,hildebrand:qnx,liedtke:construction}
aim for minimal functionality running in privileged mode.  The bare
minimum is considered to be IPC and scheduling.  Driver-level code such
as file systems, the networking stack and system call handlers run in
unprivileged mode.  Microkernel operating systems can be roughly divided
into two categories: monolithic and multiserver.

Monolithic servers contain all drivers within a single server.
This monolithic nature means that a single component failing will affect the entire
server.  Monolithic servers can be likened with full system
virtualization if the microkernel can be considered a virtual
machine monitor.  Monolithic servers share the same basic code
structure as a monolithic kernel.  From an application's perspective
the implications are much the same too: if one component in the
monolithic server fails fatally, it may affect all others as well.

Multiserver microkernels divide services into various independent
servers.  This division means that each component may fail independently.
For example, if the TCP/IP server fails, file systems may still
be usable (it depends on the application if this single failure has any
impact or not).

Part of the effort in building a microkernel operating system is coming
up with the microkernel itself and the related auxiliary functions.
The rest of the effort is implementing drivers.  A rump kernel in itself
is agnostic about how it is accessed.  As we have shown, it is possible
to use rump kernels as microkernel style servers on systems based on the
monolithic kernel.  The main idea of the anykernel is not to argue how
the kernel code should be structured, but rather to give the freedom of
running drivers in all configurations.

\subsection{Partitioned Operating Systems}

Partitioned operating systems~\cite{baumann:multikernel,wentzlaff:casefos}
run components in independent servers and use message passing for
communication between the servers.  The motivations for partitioning
the OS have to do with the modern processor architecture.  The trend
is for chips to contain an increasing number of cores, and the structure of the
chip the cores are located on starts looking like a networked system
with routing and core-to-core latency becoming real issues.  Therefore,
it makes sense to structure the OS around a networked paradigm with
explicit message-passing between servers running on massively multicore
architectures.  Since rump kernels allow hosting drivers in independent
servers, they are a viable option for the driver components of partitioned
operating systems.

One argument for partitioned operating systems is that performance
and code simplicity is improved since a single parallel thread in
a server runs until it blocks.  This means that there are no atomic
memory bus level locks in the model.  We investigated the performance
claim in Section~\ref{sect:spinlockopt} and agree with the performance
benefit.  However, based on the author's experiences with implementing
production quality file servers on top of the puffs~\cite{kantee:puffs}
cooperative multitasking model, we disagree about concurrent data
access being easier in a model without implicit thread scheduling ---
concurrency still needs to be handled.

\subsection{Plan 9}

Plan~9~\cite{pike:plan9} made the idea ``everything is a file''
reality.  Everything in Plan~9 is accessed through the file interface.
Kernel devices are driven by \textit{walking} the file name namespace
to obtain a handle and then doing read and write operations on the
handle.  In most cases the control protocol is text, although there
are some exceptions for efficiency reasons, such as the frame
buffer.  Remote resources can be attached to a process's view of
the file namespace.  The side-effect of this file-oriented operation is that distributed
computing is implied; from the application perspective
everything looks the same, regardless of whether the resource is local
or remote.  Since the communication protocol is text, there are no incompatibilities
in binary representation between various machine types.  However,
the practical implication is that the whole system must be built
from ground up to conform to the paradigm.

We showed with our remote clients that a Unix system is capable of
limited distributed operation without having to rewrite drivers or
build the entire OS from ground up.  The limitation is that the
current system call interface carries many binary subprotocols
which are incompatible between heterogeneous hosts.  Also, not all
standard system calls that applications can expect are supported,
\texttt{mmap()} being the canonical example --- memory mapping
cannot be extended to a distributed environment without the interface
contract toward the application being severed.  We do not claim
that our result is as flexible as the one offered by Plan~9, but
we do argue that it offers 98\% of the practical functionality with
2\% of the effort.

\subsection{Namespace Virtualization}

\subsubsection*{Containers}

Containers use namespace virtualization to provide multiple
different views of the operating system's namespaces to an application, \eg the file
system namespace.  These views are provided on top of one kernel instance.
Examples of the container approach include FreeBSD jails~\cite{phk:jails}, Linux
OpenVZ or the Featherweight Virtual Machine~\cite{yu:fvm}.  The effect is
the ability to present the illusion of a virtual machine to applications.
This illusion is accomplished by keeping track of which namespace each
process has access to.  When combined with the ability to virtualize
the networking stack~\cite{zec:vimage}, it is possible to run entire
disjoint guest operating systems.

The major difference to rump kernels is that in namespace virtualization
all code runs in a single kernel.  The implications are
reflected in the use cases.  Namespace virtualization can be used to
present applications a system-defined policy on the namespace they
are allowed to view.  A rump kernel as a concept is agnostic to who
establishes the policy (cf. local and microkernel clients).
Namespace virtualization cannot be substituted for any rump kernel
use case where the integrity of the kernel is in danger (\eg testing
and secure mounting of disk file systems) since damage to any
namespace can affect the entire host kernel and therefore all
virtual operating systems offered by it.

\subsubsection*{View OS}

The View OS~\cite{gardenghi:viewos} approach to namespace virtualization
is to give the application multiple choices of the entities providing
various services conventionally associated with the kernel, \eg networking
or file systems.  System calls made by processes are intercepted and
redirected to service provides.

System call interception is done using \texttt{ptrace()} or alternatively
an optimized version called \texttt{utrace()}, which requires special
host kernel support.  This approach allows the redirection policy to
exist outside of the client.  As servers, the View OS uses specially
adapted code which mimics kernel functionality, such as LWIPv6 for the
networking service.  The clientside functionality and ideology of the View
OS and virtualized lightweight servers consisting of preexisting kernel
code as offered by rump kernels can be seen to complement each other.

\subsection{Lib OS}

A library OS means a system where part of the OS functionality runs in
the same space as the application.  Initially, the library OS was meant
to lessen the amount of functionality hidden by the OS abstractions by
allowing applications low-level access to OS routines~\cite{engler:exo,
kaashoek:exoapp}.  A later approach~\cite{porter:drawbridge} used a
library as indirection to provide lightweight application sandboxing.
What is common between these approaches is that they target the
application.  We have demonstrated that it is possible, although not
most convenient, to use existing kernel code as libraries and access
functionality at a mode detailed layer than what is exposed by standard
system calls.  Our work was focused on the reuse of kernel code, and
we did not target application-level sandboxing.

\subsection{Inter-OS Kernel Code}

The common approach for moving kernel code operating system A to
operating system B is to port it.  Porting means that the original
code from A is taken, incompatible changes are made, and the code
is dropped into B.  Any improvements made by A or B will not be
compatible with B or A, respectively, and if such improvements are
desired, they require manual labor.

Another approach is a portability layer.  Here the driver
code itself can theoretically be shared between multiple systems
and differences between systems are handled by a separate portability
layer.  For example, the USB device drivers in BSD systems used to
have a portability header caller \verb+usb_port.h+.  Drivers which
were written against this header could theoretically be shared
between different BSD operating systems.  It did not work very well
in reality, and use of the macro was retired from NetBSD in 2010.
Part of the problem was that there was no established master, and
drivers were expected to flow between different operating systems,
all of which were continuously evolving.

The portability layer approach is better when the flow of code is
unidirectional, \ie all improvements are made in OS A and at a
later date transferred to OS B.  For example, the ZFS port of NetBSD
has been done this way.  New revisions of OpenSolaris could be
directly imported into NetBSD without having to merge changes.

Next, we look at various projects which provide 3rd party support
with the intent of being able to host the kernel either in or out of
the kernel on a foreign system.

\subsubsection*{OSKit}

Instead of making system A's drivers available for system B, the OSKit
project~\cite{ford:oskit} had a different motivation.
It made system A's drivers available for virtually everyone
via portability layers.  In fact, there was no single source system,
and OSKit offered drivers from multiple different operating systems
such as FreeBSD and NetBSD.  The approach OSKit took was to take
code from the source systems, incorporate it into the OSKit tree,
and produce an OSKit release.  Since OSKit functionality was not
maintained within the source systems, every update takes manual
effort.  The latest release of OSKit was in 2002.

While the anykernel architecture defines a portability layer for
kernel code, we do not claim it is as such the ideal approach for
integrating kernel code into foreign monolithic kernels.  It does
not produce a tightly integrated solution, since many subsystems
such as the memory allocator are replicated.  However, in a system
where drivers are separate servers, such as a multiserver microkernel
or a hybrid kernel, the anykernel approach with rump kernel is feasible.

\subsubsection*{Device Driver Environment (DDE)}

The DDE (Device Driver Environment) is set of patches to a guest
system which allows hosting drivers on other platforms with the
DDEKit interface~\cite{ddekitwiki}.  These concepts map to
reimplemented code in the rump kernel and the rumpuser interface,
respectively.  The purpose of DDE is to allow to running drivers as
servers, and only the lower layers of the kernel are supported.
Lower layer operation is comparable to rump kernels with microkernel
clients which call into the kernel below the system call layer.  As
an example of use, it is possible to run the unmodified Linux kernel
E1000 PCI NIC driver as userspace server on
Linux~\cite{weisbach:ddekit} with DDE/Linux.

\subsubsection*{NDIS}

NDIS stands for ``Network Driver Interface Specification''.  It is
the interface against which networking components are implemented on
the Windows operating system.  Among these components are the
network interface card (NIC) drivers, which are classified as
\textit{miniport} drivers according to Windows Driver Model.  The
Windows NIC drivers are interesting for other operating systems for
two reasons:

\begin{enumerate}
\item	There is a large number of different NICs and therefore a
	large number of NIC drivers are required.  Implementing each
	driver separately for each OS constitutes a large amount of
	work.

\item	Some drivers support devices for which no public
	documentation is available.
\end{enumerate}

The previous examples of use of foreign kernel code involved having
access to the source code.  Windows drivers are distributed only as
binary and this distribution model places additional limitations.  First, the code can
only run on the CPU architecture the driver was compiled for, namely i386 and amd64.
Second, the system hosting the foreign code must provide the same
ABI as the driver --- mere API compatibility is not enough.

The relevant part of the Windows kernel ABI, including NDIS, is
emulated by the \textit{NDIS wrapper}~\cite{ndis:fbman4}.  The
wrapper has been ported to most BSD operating systems, and enables
including binary-only Windows NIC drivers.  In addition to providing
the correct interfaces for the drivers to use, the wrapper must also
take care of adjusting the function calling convention, since BSD
operating systems use a different one from Windows.

The wrapper implementation redefines all structures used by the NDIS
interface.  It does not introduce any techniques which
could have been useful solutions when discussing the type
incompatibility challenges with running NetBSD code in a foreign host
(Section~\ref{sect:foreignhost}).  The fact that is possible to
define a function interface to attach drivers suggests that
Windows does not suffer from the inline/macro problems which affect
our ability to use standard NetBSD kernel modules on non-x86
platforms (Section~\ref{sect:abicompat}).


\subsection{Safe Virtualized Drivers}

Safe drivers in virtualized context depend on the ability to limit
driver access to only the necessary resources.  In most cases, limiting access
is straightforward.  For example, file system drivers require access
only to the backing storage.  In case of malfunction or malice,
damage will be limited to everything the driver had access to.

Hardware device drivers are typically more difficult to limit
since they perform DMA.  For example, PCI devices are bus
mastering, and DMA is done by programming the necessary physical
memory addresses to the device and letting the device handle the
I/O.  Incorrect or malicious addresses will be read or written by
the device.  Limiting DMA access is possible if the hardware includes
an IOMMU.  This capability has been used for safe unmodified virtual
device drivers~\cite{levasseur:driverreuse}.

Notably though, hardware devices do not imply problems with DMA.
We demonstrated working USB hardware device drivers.  The rump
kernel infrastructure does not currently support hardware device drivers
in the general case.  Still, there is no reason why it could not be done
on systems with IOMMU support.

\subsection{Testing and Development}

Sun's ZFS file system ships with a userspace testing library,
libzpool~\cite{zfs_source_tour}.  In addition to kernel interface
emulation routines, it consists of the Data Management Unit and
Storage Pool Allocator components of ZFS compiled from the kernel
sources.  The ztest program plugs directly to these components.
This approach has several shortcomings compared to using rump
kernels.  First, it does not include the entire file system
architecture, \eg the VFS layer.  The effort of implementing the
VFS interface (in ZFS terms the \textit{ZFS POSIX Layer}) was
specifically mentioned as the hardest part of porting ZFS to
FreeBSD~\cite{pjd:zfs}.  Therefore, it should receive non-zero
support from the test framework.  Second, the approach requires special organization
of the code for an individual driver.  Finally, the
test program is specific to ZFS.  In contrast, integrating ZFS into
the NetBSD FS-independent rump kernel test framework (described in
Section~\ref{sect:testingstudies}) required roughly 30 minutes of
work and 30 lines of code.
Using a rump kernel it is possible to test the entire driver
stack in userspace from system call to VFS layer to ZFS implementation.

\subsection{Single Address Space OS}

As the name suggests, a Single Address Space OS (SASOS) runs entirely
in a single virtual memory address space, typically 64bit, with
different applications located in different parts of the address
space.  The single system-wide address space is in contrast to the approach where each process runs
in its own address space.  Protection between applications can
be provided either by the traditional memory management
hardware approach~\cite{chase:sasosprot,heiser:mungi} or by means
of software-isolated processes (SIP)~\cite{hunt:singularity}.

A rump kernel with local clients is essentially a SASOS.  It is
possible to use multiple process contexts against the local rump
kernel with the \verb+rump_lwproc+ interfaces.  The difference to
the abovementioned SASOSs is that the runtime enforces neither fair
scheduling nor protection between these process contexts.  In other
words, everything is based on cooperation.  For our case cooperation is
a reasonable assumption, since all processes are provided by the
same binary.

Another ``single address space'' limitation we had to overcome was
having the kernel and application in a joint symbol namespace.
Any symbols exported both by the application portion
and the kernel, \eg \verb+printf()+, would cause a collision.
While there are SASOS solutions which, among other things, handle
similar symbol collision problems~\cite{deller:sasoslink}, they
require modifications to the linking and loading procedure.  Since
we wanted things to work out-of-the-box on a Unix-style system, we
opted for the renaming approach (Section~\ref{chap:rumpns}).
